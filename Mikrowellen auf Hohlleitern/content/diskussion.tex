

\section{Diskussion}
\label{sec:Diskussion}

Dieses Kapitel befasst sich mit der Diskussion der im \autoref{sec:auswertung} erhaltenen Ergebnisse.



Der Fit einer Parabel an die Messwerte der drei Schwingungsmoden aus \autoref{sec:schwingungsmoden}, wie sie in \autoref{fig:schwingmod} graphisch dargestellt ist, deckt sich sehr gut mit den Messwerten.
Die Scheitelpunkte der Parabeln liegen für kleinere Reflektorspannungen dichter beieinander, aber die größten Werte werden nicht kleiner zu niedrigeren Reflektorspannungen.
Dies kann an unsachgemäßem Ablesen der Leistung am Oszilloskop liegen, aber auch auftretenden Resonanzen innerhalb des Zusammenspiels aus Messapparaturen, die hierfür verwendet wurden.\\
Die Messung der Bandbreite war mit großen Unsicherheiten behaftet, welche vorallem bei der Abstimmempfindlichkeit sichtbar werden.
Für die Abstimmempfindlichkeit $E$ ergab sich der Wert zu:

\begin{equation}
E = (1,5 \pm 1,061) \frac{\text{MHz}}{\text{V}}
\label{eq:elbandbreite2}
\end{equation}

Die relative Unsicherheit beträgt somit ca 71\%. Dies lässt sich durch mehrere Effekte erklären. Zum einen handelt es sich hierbei um eine Einzelmessung.
Für genauere Werte sollte mehrfach gemessen und daraus der Mittelwert berechnet werden.
Zum anderen waren die Flanken der Schwingungsmode sehr viel steiler als bei einer der Parabeln, wie sie in \autoref{fig:schwingmod} zu sehen sind.
Der größte Faktor war jedoch, dass sich die Reflektorspannung nur auf 10\,V genau bestimmen ließ.\\

In \autoref{sec:fwd} wurde die Wellenlänge der verwendeten Strahlung zu $\lambda_g = (50,6 \pm 0,141)$\,mm bestimmt. Diese liegt in dem Bereich, der als Mikrowellenstrahlung bezeichnet wird.
Die Frequenz $f$ wurde auf zwei Arten bestimmt. Einmal direkt per Messung mit einem Frequenzmessgerät und einmal indirekt über die Wellenlänge und Maße des Hohlleiters.\\
Die erste Messung ergab eine Frequenz von $f = (9014 \pm 0,5)$\,MHz und die zweite $f = (8843,47 \pm 14,79)$\,MHz.
Der zweite Wert hat eine deutlich größere Unsicherheit und unterscheidet sich vom ersten um etwa 1,9\% des ersten Wertes.
Unter der Annahme, dass die erste Messmethode deutlich näher an dem realen Wert liegt, kann das Ergebnis der Zweiten auch als recht genau betrachtet werden.\\
In \autoref{fig:daempf} sind die gemessene und die Eichkurve des Dämpfungsgliedes zu sehen. 

Die Messreihe wurde bei 16,dB anstelle von 0,dB gestartet, da dort die Eichkurve des Dämpfungsgliedes leichter ablesbar war.
Der maximale Unterschied zwischen gemessener Dämpfung und abgelesener Betrug 3/,dB.
Da auch das Ablesen der Dämpfung auf der Eichkurve mit Unsicherheiten in der Größenordnung von einigen dB behaftet ist, lässt sich sagen, dass die beiden Kurven gut übereinstimmen.
Das direkte Messen der Welligkeit mittels eines SWR-Meters ist nur  für geringe Welligkeiten möglich.
Dies hat praktische Gründe:
So kommt das Messen von Welligkeiten häufig dort vor, wo stehende Wellen vermieden werden sollen und somit nur sehr geringe Welligkeiten akzeptabel sind.
Bei einer Sondentiefe von 9/,mm war somit kein Wert für das Stehwellenverhältnis mehr ablesbar.

Zur Bestimmung dieses Wertes wurden zwei verschiedene Methoden verwendet.
Bei der 3,dB-Methode ging neben den Messwerten bestimmter Stellen der Welle auch die Wellenlänge mit ein, welche selbst eine fehlerbehaftete Größe war.
Die Welligkeit ist jedoch nur proportional bzw antiproportional zu den gemessenen Werten.
Dahingegen weist die Welligkeit bei der Abschwächer-Methode einen exponentiellen Zusammenhang zu den Messwerten auf.
Dieser exponentielle Zusammenhang ist die Ursache dafür, dass kleine Schwankungen der Messwerte zu großen Schwankungen des berechneten Stehwellenverhältnisses führen und damit womöglich auch die Ursache für den großen Unterschied von ca 28\% der Ergebnisse beider Methoden.
Der durch die 3,dB-Methode erhaltene Wert $S = 9,025 \pm 0,038$ ist der genauere.

























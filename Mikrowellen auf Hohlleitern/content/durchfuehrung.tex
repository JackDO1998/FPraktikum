\section{Durchführung}
\label{sec:Durchfuehrung}
In diesem Kapitel sollen die einzelnen Schritte des Versuches erklärt werden.

\subsection{Aufbau}
\label{sec:dfaufbau}
Der Grundaufbau ist für jeden Versuchsteil der selbe. Das Reflexklystron erzeugt die Mikrowellen
diese laufen über einen ersten Hohlleiter direkt in einen Einwegrichter. Dieser sorgt mithilfe des Faradayeffektes
dafür das die Mikrowellen nur in eine Richtung nahezu verlustfrei passieren können und in die andere Richtung 
auf einen Absorber abgelenkt werden. Das sorgt dafür das das Klystron durch im Experiment entstehende 
reflektierte Mikrowellen nicht gestört wird. Darauf folgt ein Frequenzmesser, dieser verfüft über einen 
verstellbaren Resonator. Das führt dazu das wenn der Resonator in Resonanz zur durch den Hohlleiter laufenden Welle
steht diese etwas abgeschwächt wird da dann Leistung ausgekoppelt wird. Das ist auf einem Oszilloskop an einem
sogenannten Dip, also einem kleinen Einbruch in der Kurve, zu erkennen. Das darauffolgende Element ist
ein Dämpfungsglied. Mit diesem kann die Leistung durch die innere Verschiebung einer Absorberfolie
angepasst werden. Die Dämpfung ist hier maximal wenn sich die Folie im Maximum des elektrischen Feldes 
befindet. Hierauf folgen je nach Versuchsaufbau unterschiedliche Elemente. Dies kann zunächst ein Stehwellendetektor
sein, dieser besteht aus einem Stück Hohlleiter in welchem der Länge nach eine Sonde verschoben werden kann. 
Das Signal der Sonde kann über eine BNC-Buchse ausgekoppelt und gemessen werden. Die Position der Sonde
lässt sich an einer Millimeterskala ablesen. Außerdem findet ein Gleitschraubentransformator Verwendung,
in ihm kann eim Metallstift der Länge nach und in der Eindringtiefe verstellt werden. So können
Blindwiederstände, also komplexe Lasten simmuliert werden. Zum Messen des ganzen Signals im Hohlleiter
kann der Detektor verwendet werden. Dieser besteht einfach aus einer Diode welche mit einer BNC-Buchse
verbunden ist. Ein weiters Element ist der "Einstellbare Kurzschluss" dieser 
reflektiert die eintreffende Welle mit einer genau Einstellbaren Phase. Zuletzt wird noch ein Abschluss
verwendet, in ihm wird nahezu alle einfallende Leistung absorbiert. Weiter Geräte sind ein zum Klystron passendes
Netzteil, ein Oszilloskop und ein SWR-Meter. Mit dem zuletzt genannten Gerät lässt sich das Stehwellenverhältnis
messen.

\subsection{}
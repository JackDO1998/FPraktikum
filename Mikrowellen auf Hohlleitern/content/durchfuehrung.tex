\section{Durchführung}
\label{sec:Durchfuehrung}
In diesem Kapitel sollen die einzelnen Schritte des Versuches erklärt werden.

\subsection{Aufbau}
\label{sec:dfaufbau}
Der Grundaufbau ist für jeden Versuchsteil der selbe. Das Reflexklystron erzeugt die Mikrowellen
diese laufen über einen ersten Hohlleiter direkt in einen Einwegrichter. Dieser sorgt mithilfe des Faradayeffektes
dafür das die Mikrowellen nur in eine Richtung nahezu verlustfrei passieren können und in die andere Richtung 
auf einen Absorber abgelenkt werden. Das sorgt dafür das das Klystron durch im Experiment entstehende 
reflektierte Mikrowellen nicht gestört wird. Darauf folgt ein Frequenzmesser, dieser verfüft über einen 
verstellbaren Resonator. Das führt dazu das wenn der Resonator in Resonanz zur durch den Hohlleiter laufenden Welle
steht diese etwas abgeschwächt wird da dann Leistung ausgekoppelt wird. Das ist auf einem Oszilloskop an einem
sogenannten Dip, also einem kleinen Einbruch in der Kurve, zu erkennen. Das darauffolgende Element ist
ein Dämpfungsglied. Mit diesem kann die Leistung durch die innere Verschiebung einer Absorberfolie
angepasst werden. Die Dämpfung ist hier maximal wenn sich die Folie im Maximum des elektrischen Feldes 
befindet. Hierauf folgen je nach Versuchsaufbau unterschiedliche Elemente. Dies kann zunächst ein Stehwellendetektor
sein, dieser besteht aus einem Stück Hohlleiter in welchem der Länge nach eine Sonde verschoben werden kann. 
Das Signal der Sonde kann über eine BNC-Buchse ausgekoppelt und gemessen werden. Die Position der Sonde
lässt sich an einer Millimeterskala ablesen. Außerdem findet ein Gleitschraubentransformator Verwendung,
in ihm kann eim Metallstift der Länge nach und in der Eindringtiefe verstellt werden. So können
Blindwiederstände, also komplexe Lasten simmuliert werden. Zum Messen des ganzen Signals im Hohlleiter
kann der Detektor verwendet werden. Dieser besteht einfach aus einer Diode welche mit einer BNC-Buchse
verbunden ist. Ein weiters Element ist der "Einstellbare Kurzschluss" dieser 
reflektiert die eintreffende Welle mit einer genau Einstellbaren Phase. Zuletzt wird noch ein Abschluss
verwendet, in ihm wird nahezu alle einfallende Leistung absorbiert. Weiter Geräte sind ein zum Klystron passendes
Netzteil, ein Oszilloskop und ein SWR-Meter. Mit dem zuletzt genannten Gerät lässt sich das Stehwellenverhältnis
messen.

\subsection{Untersuchung der Moden}
\label{sec:moden}
Um die Moden des Reflexklystrons zu untersuchen wird hinter dem Reflexklystron mit dem Einwegrichter, dem 
Frequenzmesser und dem Abschwächer der Detektor montiert und über ein Koaxialkabel mit dem Eingang
für die Vertikalablenkung des Oszilloskopes verbunden. Die Horizontalablenkung wird mit dem Netzgerät des Klystrons
verbunden. Die Abszisse repräsentiert nun die Reflektorspannung während die Ordinate auf dem Oszilloskop
die empfangene Leistung repräsentiert. Es ist nun für jeden Schwingungsmodus des Klystrons ein Berg zu sehen.
Nun wird der Frequenzmesser solange verstellt bis in der jeweiligen Spitze des Berges ein Dip zu sehen ist.
Dann kann die Mittelfrequenz dieser Resonanz vom Frequenzzähler abgelesen werden. Nun wird die Reflektorspannung
verstellt und so die Spannungen gemässen bei denen die Schwingung einsetzt bzw. wieder aussetzt. Dies wird für mehrere
Moden wiederholt. Im weiteren werden die "Punkte halber Leistung" auf der linken und rechten Flanke des 
Resonanzberges notiert.

\subsection{Messung von Frequenz, Wellenlänge und Dämpfung}
\label{sec:frequenzmessung}
Für diesen Versuchsteil wird nach dem Frequenzmesser der Stehwellendetektor und daran der Abschluss montiert.
An den Stehwellendetektor wird mittels Koaxialkabel das SWR-Meter angeschlossen. Nun wird die Reflektorspannung
so eingestellt das sich am SWR-Meter ein Maximalauschlag ergibt. Dann wird der Frequenzzähler so eingestellt das 
der entstehende Dip auf dem SWR-Meter zu sehen ist und die Frequenz wird notiert. Um nun die Wellenlänge zu bestimmen
muss der Abschluss durch den einstellbaren Kurzschluss ersetzt werden. Jetzt wird die Sonde des Stehwellendetektors
solange verschoben bis am SWR-Meter ein mimimaler Ausschlag zu erkennen ist. Die Sondenposition wird notiert und die
Sonde bis zum nächsten Minimum verschoben. Die Wellenlänge ergibt sich als Doppeltes des Abstandes der beiden 
Minima. Die Dämpfungskurve wird nun ausgemessen indem die Verstärkung am SWR-Meter soweit aufgedreht wird das 
der Zeiger zum Vollauschlag kommt. Dann wird Die Dämpfung solange erhöt bis das Messgerät 2dB zeigt. Dieser Schritt
wird bis zu einer Dämpfung von 10 dB wiederholt und jedesmal wird die Position der Millimeterschraube des 
Dämpfungsgliedes notiert.

\
\section{Theorie} 
\label{sec:Theorie}

In diesem Kapitel werden die theoretischen Hintergründe dieses Versuches erläutert. Dabei wird insbesondere auf die in der Durchführung verwendeten Schaltungen eingegangen.

\subsection{Mikrowellen allgemein}
\label{sec:thallgemein}
Mikrowellen sind je nach definition Wellen mit Wellenlängen mit Wellenlängen von etlichen zentimetern bis
zu Bruchteilen von millimetern. Das entspricht Frequenzen von etwa 300 MHz bis 1 THz. In diesem Versuch werden
Wellen im bereich von wenigen GHz verwendet. Das entspricht Wellenlängen von wenigen zentimetern und hat den
Grund das für solche Wellenlängen passende Hohlleiter gut zu handhaben sind.

\subsection{Das Klystron}
\label{sec:thklystron}
Zur erzeugung der Mikrowellen kann ein Klystron verwendet werden.
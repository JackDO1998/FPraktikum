\section{Theorie} 
\label{sec:Theorie}

In diesem Kapitel werden die theoretischen Hintergründe dieses Versuches erläutert. Dabei wird insbesondere auf die in der Durchführung verwendeten Schaltungen eingegangen.

\subsection{Mikrowellen allgemein}
\label{sec:thallgemein}
Mikrowellen sind je nach definition Wellen mit Wellenlängen mit Wellenlängen von etlichen zentimetern bis
zu Bruchteilen von millimetern. Das entspricht Frequenzen von etwa 300 MHz bis 1 THz. In diesem Versuch werden
Wellen im bereich von wenigen GHz verwendet. Das entspricht Wellenlängen von wenigen zentimetern und hat den
Grund das für solche Wellenlängen passende Hohlleiter gut zu handhaben sind.

\subsection{Das Klystron}
\label{sec:thklystron}
Zur erzeugung der Mikrowellen kann ein Klystron verwendet werden. Ein Klystron besteht im allgemeinen 
aus zwei Hohlraumresonatoren und einer dazwischen gelegenen Driftstrecke. In den ersten Hohlraumresonator 
wird ein Schwingung mit passender Frequenz eingekoppelt, dann wird ein Elektronenstrahl eingeführt, welcher 
von der Schwingung im Resonator Moduliert wird. Das heißt Teilchen werden, da das elektrische Feld ortsabhängig
ist, ortsabhängig beschleunigt oder abgebremst. So bilden sich über die Driftstracke 
einzelne Elektronenpakete die Bunches genannt werden. Diese Bunches erzeugen im zweiten Hohlraumresonator eine
zur im ersten Resonator eingekoppelten Schwingung, bedeutend stärkere Schwingung. Hiervon kann ein kleiner Teil 
ausgekoppelt werden und für das Experiment vervendet werden. In deiesem Versuch wird jedoch kein klassisches
Klystron sondern das sogenannte Reflexklystron verwendet. Die besonderheit ist hier das es nur einen Resonator 
gibt in welchem die Schwingung sowohl ein als auch ausgekoppelt wird. Dazu wird der Elektronenstrahl am Ender der 
Driftstrecke von einer Kathode reflektiert und wieder in den Resonator geleitet wo dann wieder eine Schwingung
erzeugt wird.


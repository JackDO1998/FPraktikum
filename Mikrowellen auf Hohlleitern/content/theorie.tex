\section{Theorie} 
\label{sec:Theorie}

In diesem Kapitel werden die theoretischen Hintergründe dieses Versuches erläutert. Dabei wird insbesondere auf die in der Durchführung verwendeten Schaltungen eingegangen.

\subsection{Mikrowellen allgemein}
\label{sec:thallgemein}
Mikrowellen sind je nach definition Wellen mit Wellenlängen mit Wellenlängen von etlichen zentimetern bis
zu Bruchteilen von millimetern. Das entspricht Frequenzen von etwa 300 MHz bis 1 THz. In diesem Versuch werden
Wellen im bereich von wenigen GHz verwendet. Das entspricht Wellenlängen von wenigen zentimetern und hat den
Grund das für solche Wellenlängen passende Hohlleiter gut zu handhaben sind.

\subsection{Das Klystron}
\label{sec:thklystron}
Zur erzeugung der Mikrowellen kann ein Klystron verwendet werden. Ein Klystron besteht im allgemeinen 
aus zwei Hohlraumresonatoren und einer dazwischen gelegenen Driftstrecke. In den ersten Hohlraumresonator 
wird ein Schwingung mit passender Frequenz eingekoppelt, dann wird ein Elektronenstrahl eingeführt, welcher 
von der Schwingung im Resonator Moduliert wird. Das heißt Teilchen werden, da das elektrische Feld ortsabhängig
ist, ortsabhängig beschleunigt oder abgebremst. So bilden sich über die Driftstracke 
einzelne Elektronenpakete die Bunches genannt werden. Diese Bunches erzeugen im zweiten Hohlraumresonator eine
zur im ersten Resonator eingekoppelten Schwingung, bedeutend stärkere Schwingung. Hiervon kann ein kleiner Teil 
ausgekoppelt werden und für das Experiment vervendet werden. In deiesem Versuch wird jedoch kein klassisches
Klystron sondern das sogenannte Reflexklystron verwendet. Die besonderheit ist hier das es nur einen Resonator 
gibt in welchem die Schwingung sowohl ein als auch ausgekoppelt wird. Dazu wird der Elektronenstrahl am Ender der 
Driftstrecke von einer Kathode reflektiert und wieder in den Resonator geleitet wo dann wieder eine Schwingung
erzeugt wird. Optimale verstärkungen werden erzeugt wenn die ein und augekoppelte Welle Phasengleich sind. 
Die Phasenverschiebung lässt sich über die länge der Driftstrecke steuern. Die Länge der Driftstrecke kann 
sowohl durch mechanische verschiebung der reflektierenden Kathode als auch durch ihre Spannung gegenüber 
der Anode verändert und eingestellt werden. Die optimalen Verstärkungen liegen bei Transitzeiten von
$\tau=T_0(n+\frac{3}{4})$ mit $n \in \mathbb{N}$ und $T_0$ der Periodendauer der eingekoppelten Schwingung.

\subsection{Der Hohlleiter}
\label{sec:thwellenleiter}
Ein Hohlleiter für Elektromagnetische Wellen ist im allgemeinen ein Rohr aus einem Material welches in der 
Lage ist elektrischen Strom zu leiten. Er dient dazu Energie in Form von EM-Wellen gebündelt zu leiten und
eine Ausbreitung in alle Raumrichtungen einzuschränken. Der Querschnitt eines Hohlleiters ist im Grunde 
beliebig. In der Praxis werden jedoch aufgrund der vergleichsweise einfachen berechenbarkeit meist rechteckige
runde oder allenfalls ovale Querschnitte gewählt. In diesem Versuch werden Hohlleiter mit einem rechteckigen
Querschnitt verwendet. Jeder Hohlleiter hat eine sogenannte Cutoff Wellenlänge $\lambda_c$ das ist die maximale
Wellenlänge also minimale Frequenz welche die eingekoppelte Welle haben darf um noch übertragen zu werden.
Sie ist bei der rechteckigen Variante abhängig von der Breite $a$ des Hohlleiters, $\lambda_c=2a$ an diesem Punkt
passt genau noch eine Halbwelle zwischen die Seitenwände des Hohlleiters. Im Hohlleiter können sich zwei verschiedene
Grundmoden ausbreiten, einerseits die "transversal elektrische" TE-Mode bei der das elektrische Feld eine Komponente
in Ausbreitungsrichtung auweist und andererseits die "transversal magnetische" TM-Mode bei welcher das Magnetfeld
eine Komponente in Ausbreitungsrichtung aufweist. Die Wellen werden mit $H_{nm}$ bzw. $E_{nm}$
bezeichnet. n steht hier für die Anzahl der Wellenmaxima in horizontaler Richtung während m für die Anzahl der 
Wellenmaxima in Vertikaler Richtung steht.


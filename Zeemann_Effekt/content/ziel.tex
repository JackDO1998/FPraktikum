\section{Zielsetzung}
\label{sec:ziel}
Das Ziel des Versuchs ist die Überprüfung des Zeeman-Effekts bei niedrigen Flussstärken am Beispiel von Cadmium.\\
Der Zeeman-Effekt beschreibt die Aufspaltung der Energienieveaus eines Atoms unter Einfluss eines konstanten externen Magnetfeldes.
Bei niedrigen Flussstärken wird zwischen normalen und anormalen Zeeman-Effekt unterschieden.
Beide Effekte werden in diesem Versuch mittels der Spektrallinien-Aufspaltung an Cadmium untersucht.
Dabei wird auch die Polarisation der emittierten Strahlung berücksichtigt.

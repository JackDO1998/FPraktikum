\section{Durchführung}
\label{sec:Durchfuehrung}
In diesem Kapitel sollen die einzelnen Schritte des Versuches erklärt werden.
\subsection{Ausrichtung und Start des Laserbetriebs}
\label{sec:ausrichtung}
Um den Laser zur Funktion zu bringen müssen alle Komponenten nach einer Strahlachse ausgerichtet werden.
Dazu wird ein grüner Hilfslaser am einen ende der optischen Bank eingeschaltet und auf ein Fadenkreuz 
am anderen Ende ausgerichtet. Nun werden nacheinander das Laserrohr und die Resonatorspiegel auf der
Bank arretiert und so ausgerichtet das die entsprechenden Beugungsringe und Reflexe wieder ins Fadenkreuz 
treffen. Die Resonatorspiegel werden dazu zunächst sehr nah an den Enden des Laserrohres montiert.
Sobald alle komponenten gut ausgerichtet sind wird der grüne Laser abgeschaltet und die Pumpspannung des 
HeNe-Lases eingeschaltet. Das Laserrohr beginnt nun rötlich zu leuchten. Nach kurzem nachstellen der 
Resonatorspiegel stellt sich der Laserbetrieb ein was an einem roten Lichtpunkt im Fadenkreuz zu erkennen ist.

\subsection{Überprüfung der Stabilitätsbedingung}
\label{sec:stabilitaetsbedingung}
Um die Stabilitätsbedingung zu überprüfen wird zunächst die von den Resonatorspiegel abhängige maximale 
Länge berechnet unter welcher der Laser noch stabil läuft. Im Anschluss daran werden die beiden Spiegel 
schrittweise voneinander entfernt und immer wieder neu ausgerichtet um den Laserbetrieb aufrecht zu erhalten.
Dabei wird jedes mal mit Hilfe einer Photodiode die maximale Leistung eingestellt.
Wenn der Laserbetrieb bei der maximal möglichen Länge noch aufrecht erhalten werden kann ist die 
Stabilitätsbedingung sicher erfüllt. Dies wird mit zwei verschiedenen Spiegeln durchgeführt.

\subsection{Beobachtung von TEM-Moden}
\label{sec:temmoden}
In der Strahlachse des Lasers wird eine Streulinse und dahinter eine orthogonal zur Strahlachse in der 
horizontalen Ebene bewegliche Photodiode positioniert. Der Laser läuft zunächst in der TEM$_{00}$-Mode. 
Die zugehörige Intensitätsverteilung kann vermessen werden indem die Photodiode in kleinen Schritten durch 
den aufgeweiteten Laserpunkt bewegt wird. Nun wird zwischen einem Resonatorspiegel und dem Laserrohr ein
dünner Wolframdraht positioniert. So können weitere TEM-Moden beobachtet werden. Diese werden auf die gleiche
Weise vermessen.

\subsection{Bestimmung der Polarisation}
\label{sec:polarisation}
Um die Polarisation des Laserlichtes zu vermessen wid ein drehbahrer Polarisationsfilter in den Strahlweg 
gebracht. Hinter dem Polarisationsfilter wird wieder dei Photodiode positioniert. Nun wird der
Polarisationsfilter in kleinen Schritten gedreht und nach jeder Drehung die Intensität gemessen.

\subsection{Frequenzspektrum}
\label{sec:Frequenzspektrum}
Der Laser wird bei unterschiedlichen Resonatorlängen auf eine zeitlich hochauflösende Photodiode gerichtet.
Die Photodiode ist an einen Spektrum-Analyzer angeschlossen. Hier können die Frequenzen der entstehenden 
Schwebungen als Peaks abgelesen werden.

\subsection{Bestimmung der Wellenlänge}
\label{sec:wellenlaenge}
Um die Wellenlänge des Lasers zu bestimmen wid in den Strahlgang des Lasers ein optisches Gitter gestellt.
Auf einem Schirm können nun als Resultat der Frauenhoferbeugung mehere Strahlungsmaxima beobachtet werden.
Aus ihren Abständen kann die Wellenlänge des Laseers berechnet werden.
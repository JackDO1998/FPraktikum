

\section{Diskussion}
\label{sec:Diskussion}

Dieses Kapitel befasst sich mit der Diskussion der im \autoref{sec:auswertung} erhaltenen Ergebnisse.
In \autoref{sec:linearverstaerker} liegen die Steigungen der an die Plateaus angepassten Funktionen
bei Null oder werden von ihrem eigenen Fehler betragsmäßig um ein Vielfaches übertroffen. Dies entspricht
absolut der theoretischen Erwartung. Die theoretischen Verstärkungsafaktoren liegen bei $x_{1t}=100$ und
$x_{2t}=10$. Die experimentell gemessenen entsprechen etwa dem Y-Achsenabschnitt der Ausgleichsgraden aus
\autoref{fig:linearverstaerker} und liegen bei $x_{1e}=(85.819\pm 2.946)$ und $x_{2e}=(9.834 \pm 0.095)$.
Das entspricht Abweichungen von $\SI[]{14.818}[]{\%}$ und $\SI[]{1.66}[]{\%}$. Diese Fehler entstehen vermutlich
dadurch das es sich nicht um einen Idealen Operationsverstärker mit unendlichem Eingangs- und null 
Ausgangswiderstand handelt, sondern um eine reale Schaltung mit hohem Eingangs und geringem Ausgangswiderstand
der die optimalwerte nicht erreicht.

In \autoref{sec:schmitt} wurde der Kippunkt des Schmitttriggers gemessen.





Mögliche Fehler können duuch schlechte Messwerte des Oszilloskops erklärt werden. Das Oszilloskop misst 
die Spannung immer von der untersten zur obersten Signalspitze. Diese Spannungen liegen oft weit oberhalb 
der theoretisch messbaren Werte welche durch die Einganspannung beschränkt sind. Sie sind vermutlich auf 
die interne Schaltung des Operationsverstärkers zurückzuführen und würden in einem zum Beispiel
durch ein Niederohmiges Messgeräten belasteten Stromkreis vermutlich nicht auftreten. Zudem kann es schwirig
sein das exakte Maximum eines Signals zu finden um den zeitlichen Abstand zu einem anderen Maximum zu bestimmen.
Das kann zu Fehlern in der Phasenverschiebung fürhen. Zudem waren die Ausgangsignale des Operationsverstärkers
oft mit einem hochfrequenten Rauschen überlagert was bei der exakten vermessung der Signale zu Problemen führt.

























\section{Durchführung}
\label{sec:Durchfuehrung}
In diesem Kapitel sollen die einzelnen Schritte des Versuches erklärt werden.
\subsection{Der invertierende Linearverstärker}
Für zwei verschiedene Widerstandsverhältnisse wird die Frequenzabhängigkeit 
des Verstärkungsfaktors sowie der Phasenverschiebung untersucht. Dazu werden ein- 
und Ausgansspannung der Schaltung gemessen und am Oszilloskop die Phasenverschiebung abgelesen.

\subsection{Der Umkehrintegrator und der invertierender Differenzierer}
Bei diesen beiden Schaltungen wird das verhalten bei verschiedenen Einganssignalformen untersucht.
Dazu wird von einem Signalgenerator zunächst ein Sinussignal, dann eine Rechteckspannung und anschließend
eine Dreiecksspannung erzeugt und diese als Einganssignal für die jeweilige Schaltung genutzt.
Auf eienm Oszilloskop wird dann das Ausgangsignal dargestellt.
Zudem wird auch hier das Frequenzverhalten des Verstärkungsfaktors untersucht.

\subsection{Der Schmitt-Trigger}
Für die Untersuchung des Schmitttriggers wird der Operationsverstärker als Schwellwertschalter
beschaltet. Durch almähliches erhöhen der Amplitude des eingespeisten Snussignals wird durch gleichzeitige
Spannungsmessung der Kippunkt ermittelt an welchem der Trigger durchschaltet und sperrt.

\subsection{Der Signalgenerator}
Beim Signalgenerator wederden Schmitttrigger und Umkehrintegrator hintereinander geschaltet und 
die Signalformen werden genauer untersucht.
\section{Theorie}
\label{sec:Theorie}
In diesem Kapitel sollen kurz die theoretischen Grundlagen des HeNe-Lasers erleutert werden.

\subsection{Aufbau und Funktion eines Lasers}
\label{sec:aufbauUndFunktion}
Das Wort Laser ist ein Akronym und steht für Light Amplification by Stimulated Emission of Radiation.
Jeder Laser besteht aus einem optischen Resonator, einem verstärkendem Medium und einer Energiepumpe.
Der optische Resonator besteht an sich aus zwei gegenüberliegenden, hochreflektiven Spiegeln von 
denen mindestens einer halbdurchlässig ist. Die Spiegel können sowohl planar als auch konkav gebaut sein.
Zwischen den Siegeln, also inneralb des optischen Resonators befindet sich das verstärkende Medium. Hier 
wird das Laserlicht sowohl erzeugt als auch durch die später noch erklärte stimmulierte Emmission
verstärkt. Beim Helium-Neon-Laser ist das verstärkende Medium das Neon. Da das Experiment bei Raumtemperatur 
unter nicht zu hohen Drücken durchgeführt wird, liegt es wie auch das Helium als Gas in einem Kapilarrohr vor.
Das letzte zur Funktion nötige Element ist die Energiepumpe, sie liefert die Energie nach welche über das 
ausfallende Laserlicht und an diversen anderen Stellen aus dem System entkommt. Im Fall des HeNe-Lasers
ist das Helium die Energiepumpe. Es wird durch zwei Elektroden an welchen eine Hochspannung anliegt
auf einen höheres Energieniveau gebracht und gibt seine Energie dann später über thermische Stöße an das 
Neon ab. Der Laser funktioniert also dadurch das Neon durch das angeregte Helium auf einen höheren 
Energiezustand gebracht wird, das Neon beginnt zu leuchten. Kohärent und linear polarisiert wird das Licht
durch den Resonator und Brewsterfenster.

\subsection{Brewsterfenster}
\label{sec:Brewsterfenster}
Die Brewsterfenster sind ein optionales Bauteil, werden in diesem Versuch jedoch verwendet.
Sie bestehen aus planparallelen Glaspaltten welche zur Strahlachse im Brewsterwinkel angeordnet sind.
Sie reflektieren das senkrecht zur Einfallsebene polarisierte Licht und lassen das parallel polarisierte 
Licht vollständig durch. Dadurch wird die senkrechte polarisation stark unterdrückt.

\subsection{Emmission und Absorbtion}
\label{sec:EmmissionUndAbsorbtion}
Zunächst wird ein System betrachtet welches nur zwei Energiezustände, $n_1$ und $n_2$ kennt.
Es gibt nun drei mögliche Prozesse welchen einen wechsel des Energiezustandes zur folge haben.
Der erste ist die Absorbtion. Das System absorbiert ein Energiequant und gelangt so vom nicht 
angeregten Zustand $n_1$ in den angeregten Zustand $n_2$. Der zweite Prozess ist die spontane
Emmission. Hier befindet sich das System zunächst in $n_2$ und fällt dann zu einem unbestimmten 
Zeitpunkt unter Emmission eines in Raumrichtung und Phase unbestimmten Energiequants auf $n_1$ zurück.
Der dritte und für den Laser entscheidende Prozess ist die stimmulierte Emmission. Hier befindet sich 
das System ebenfalls in $n_2$ wird dann aber von einem Photon getroffen und fällt unter Emmission
eines dem einfallenden Photon in Raumrichtung und Phase gleichenden Photons zurück auf $n_1$. Dieser letzte
Prozess führt dazu das dass Laserlicht nur in Richtung der Strahlachse verstärkt wird, kohärent ist, und
falls Brewsterfenster verwendet werden, auch linear polarisiert ist. Die einfallenden Energiequante müssen 
energetisch jeweils dem Energieunterschied von $n_1$ und $n_2$ entsprechen.


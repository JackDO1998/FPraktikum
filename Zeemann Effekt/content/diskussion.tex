

\section{Diskussion}
\label{sec:Diskussion}

Dieses Kapitel befasst sich mit der Diskussion der im \autoref{sec:auswertung} erhaltenen Ergebnisse.
In \autoref{tab:abweichungen} werden die berechneten mit den experimentell gefundenen Landé-Faktoren verglichen
und die entsprechende Abweichung nach \autoref{eq:prozentuale} berechent.

\begin{table}
    \centering
      \caption{Abweichung der Landé-Faktoren.}
      \label{tab:abweichungen}
      \sisetup{table-format=1.2}
      \begin{tabular}{S[table-format=3.2] S S S S [table-format=3.2]}
        \toprule
        {$\lambda/\si[]{\nano \metre}$} &{Übergang} & {$g_{theo}$}&{$g_{j}$}&{Abweichung / \%}\\
        \midrule
        {$648.8$}&{$\sigma$}&{$1.0$}&{$209.24\pm 29.86$}&{$209.24$}\\
        {$480.0$}&{$\sigma$}&{$1.5$}&{$8.89\pm 0.7$}&{$492.7$}\\
        {$480.0$}&{$\pi$}&{$0.5$}&{$21.34\pm 0.44 $}&{$5168$}\\
        \bottomrule
      \end{tabular}
    \end{table}

Die Abweichungen sind unter anderem damit zu erklären das das B-Feld für den annormalen Zeemanneffekt zu klein war
da der Elektromagnet keine größere Leistung bringen konnte. Das führt dazu das bei dem blauen Pi-Übergang nahezu keine
Aufspaltung der Linien erkennbar ist. Hier kann nur eine verbreiterung beobachtet werden. Das wird noch von der Tatsache
das der polarisationsfilter nicht ideal ist, die dritte Linie also jeweils nur abschwächt und nicht vollständig herausfiltert
verstärkt. Ein weiters Problem dürfte sein das insbesondere bei den blauen Linien eine Vielzahl an Moden aufgenommen wurde,
das führt dazu das jede einzelne Mode weniger Pixel breit ist und sich somit Linien schlechter unterscheiden lassen.

\section{Literatur}
[1] V27 Der Zeemanneffekt, TU Dortmund



\section{Auswertung}
\label{sec:auswertung}

In diesem Kapitel werden die aufgenommenen Messwerte ausgewertet.
\subsection{Vorbereitung}
\label{sec:vorbereitunf}
\subsubsection{Luhmer-Gehrcke Platte}
\label{sec:luhmer}
Die Eigenschaften der Luhmer-Ghercke Platte lassen sich mit den Materialeigenschaften
der Platte bestimmen. Die in diesem Versuch verwendete Luhmer Ghercke Platte hat die
Maße $d=\SI[]{4}[]{mm}$, $L=\SI[]{120}[]{mm}$. Die beiden Spektrallinien welche betrachtet werden sollen
sind:
\begin{center}
  $\lambda_{rot}=\SI[]{643.8}[]{nm}$ und
  $\lambda_{blau}=\SI[]{480.0}[]{nm}$.
\end{center}
Für diesen Versuchsaufbau ergeben sich die Wellenlängenabhängigen Brechungsindizes:
\begin{center}
  $n_{rot}=1.4567$ und
  $n_{blau}=1.4635$.
\end{center}
Mit diesen Angaben kann dann über \autoref{eq:aufloesungsvermoegen} das Auflösungsvermögen $A$ und über
\autoref{eq:dispersionsgebiet} das Dispersionsgebiet $\Delta\lambda$ berechnet werden. Die Ergebnisse
sind in \autoref{tab:LGPlatte} dargestellt.
\begin{table}
  \centering
    \caption{Wellenlängenabhängige Werte der Lummer-Gehrke Platte.}
    \label{tab:LGPlatte}
    \sisetup{table-format=1.2}
    \begin{tabular}{S[table-format=3.2] | S S S [table-format=3.2]}
      \toprule
      {Größe}&{$\SI[]{643.8}[]{nm}$} & {$\SI[]{480.0}[]{nm}$}\\
      \midrule
      {$$A$$}&{$$209128.59$$}&{$$285458.06$$}\\
      {$\Delta \lambda / \si[]{\pico\metre} $}&{$$48.91$$}&{$$26.95$$}\\
      \bottomrule
    \end{tabular}
  \end{table}

  \subsubsection{Bestimmung der Landé-Faktoren}
  Die Unterschiede in der Aufspaltung der Spektrallinien ist abhängig von den Landé
Faktoren $g_j$. Für den roten Übergang zwischen den Niveaus $^1 P_1 \leftrightarrow  ^1D_2$ und den blauen
Übergang zwischen den Niveaus $^3 S_1 \leftrightarrow  ^3P_1$ ergeben sich für die einzelnen Landé Faktoren,
mit \autoref{eq:lande} die Werte in \autoref{tab:lande}. Mit der Notation $^{2𝑆+}1𝐿_𝑗$ für die Niveaus, wobei
$L$ der Name des Niveaus ist, welchem ein fester Drehimpuls zugeordnet ist, ergibt sich
\begin{table}
  \centering
    \caption{Berchnung der Landé-Faktoren.}
    \label{tab:lande}
    \sisetup{table-format=1.2}
    \begin{tabular}{S[table-format=3.2] | S S S S S [table-format=3.2]}
      \toprule
      {Niveau}&{J}&{S}&{L}  & {$g_j$}\\
      \midrule
      {$ ^1 P_1$}&{1}&{0}&{1}&{$1$}\\
      {$ ^1 D_2$}&{2}&{0}&{2}&{$1$}\\
      {$ ^3 S_1$}&{1}&{1}&{0}&{$2$}\\
      {$ ^3 P_1$}&{1}&{1}&{1}&{$\frac{2}{2}$}\\
      
      \bottomrule
    \end{tabular}
  \end{table}
\subsection{Vermessung des Elektromagneten}
Es ist aufgrund des Versuchsaufbaus nicht möglich die magnetische Flussdichte $B$ zwischen den 
beiden Polschuhen des Elektromagneten zu bestimmen während die Cadmiumdampflampe eingeführt ist.
Daher muss das Magnetfeld vorher mittels einer Hallsonde in abhängigkeit vom Spulenstrom ausgemessen
werden. Wenn die Cadmiumdampflampe dann eingeführt ist muss nurnoch der passende Spulenstrom 
eingestellt werden. In \autoref{tab:magnetfeld} sind die Messdaten für die abfallende Seite der
Hysteresekurve dargestellt.
\begin{table}
    \centering
      \caption{In der Tabelle sind die Messdaten für den Spulenstrom $I$ und die resultierende Flussdichte $B$ dargestellt.}
      \label{tab:magnetfeld}
      \sisetup{table-format=1.2}
      \begin{tabular}{S[table-format=3.2] S S [table-format=3.2]}
        \toprule
        {$I$/[$\si[]{A}$]} & {$B$/[$\si[]{mT}$]}\\
        \midrule
        {$$5.0$$} & {$$ 452.1$$}\\
        {$$4.8$$} & {$$	440.4$$}\\
        {$$4.6$$} & {$$	430.2$$}\\
        {$$4.4$$} & {$$	415.4$$}\\
        {$$4.2$$} & {$$	403.4$$}\\
        {$$4.0$$} & {$$	388.0$$}\\
        {$$3.8$$} & {$$	371.7$$}\\
        {$$3.6$$} & {$$	356.7$$}\\
        {$$3.4$$} & {$$	338.8$$}\\
        {$$3.2$$} & {$$	320.9$$}\\
        {$$3.0$$} & {$$	305.5$$}\\
        {$$2.8$$} & {$$	288.2$$}\\
        {$$2.6$$} & {$$	266.8$$}\\
        {$$2.4$$} & {$$	248.8$$}\\
        {$$2.2$$} & {$$	229.6$$}\\
        {$$2.0$$} & {$$	209.2$$}\\
        {$$1.6$$} & {$$	169.8$$}\\
        {$$1.2$$} & {$$	131.1$$}\\
        {$$0.8$$} & {$$	89.4$$}\\
        {$$0.4$$} & {$$	50.7$$}\\
        {$$0.0$$} & {$$	 9.9$$}\\
        
        \bottomrule
      \end{tabular}
    \end{table}

    Die Daten aus \autoref{tab:magnetfeld} wurden in \autoref{fig:magnetfeld} dargestellt. Zudem wurde 
    an die Daten ein Polynom dritten Grades angepasst die verwendeten Parameter lauten:
    \begin{center}
        $$a_3 = -0.00105 \pm 0.00008$$\\
        $$a_2 =  0.00359 \pm 0.00062$$\\
        $$a_1 =  0.09673 \pm 0.00135$$\\
        $$a_0 =  0.01059 \pm 0.00080$$
        
    \end{center}

    \begin{figure}
        \centering
        \includegraphics[width=1\textwidth]{content/grafiken/magnetfeld.pdf}
        \caption{Magnetische Flussdichte des verwendeten Elektromagneten in Abhängigkeit des Spulenstroms.}
        \label{fig:magnetfeld}
      \end{figure}

      \begin{table}[t]
\centering
\caption{Daten aus:rot mit magnet 90 (2).}
\label{tab:rot mit magnet 90 (2)}
\sisetup{table-format=1.2}
\begin{tabular}{S[table-format=3.2] S S S [table-format=3.2]}
\toprule
{Linie Nr.} & {Breite/[$\si[]{px}$]}&{Helligkeit}\\
\midrule
{$$0$$}&{$$40$$}&{$$17.05$$}\\
{$$1$$}&{$$57$$}&{$$19.98$$}\\
{$$2$$}&{$$58$$}&{$$20.95$$}\\
{$$3$$}&{$$58$$}&{$$21.81$$}\\
{$$4$$}&{$$60$$}&{$$21.18$$}\\
{$$5$$}&{$$64$$}&{$$20.53$$}\\
{$$6$$}&{$$62$$}&{$$19.52$$}\\
{$$7$$}&{$$65$$}&{$$18.57$$}\\
{$$8$$}&{$$64$$}&{$$17.73$$}\\
{$$9$$}&{$$69$$}&{$$16.94$$}\\
{$$10$$}&{$$71$$}&{$$15.97$$}\\
{$$11$$}&{$$53$$}&{$$15.49$$}\\
{$$12$$}&{$$56$$}&{$$14.2$$}\\
{$$13$$}&{$$38$$}&{$$13.68$$}\\
\midrule
{$$\diameter$$}&{$$58.21\pm2.55$$}&{$$18.12\pm0.71$$}\\
\bottomrule
\end{tabular}
\end{table}

      \begin{table}[t]
\centering
\caption{Daten aus:rot mit magnet 90 (2).}
\label{tab:rot mit magnet 90 (2)}
\sisetup{table-format=1.2}
\begin{tabular}{S[table-format=3.2] S S S [table-format=3.2]}
\toprule
{Linie Nr.} & {Breite/[$\si[]{px}$]}&{Helligkeit}\\
\midrule
{$$0$$}&{$$40$$}&{$$17.05$$}\\
{$$1$$}&{$$57$$}&{$$19.98$$}\\
{$$2$$}&{$$58$$}&{$$20.95$$}\\
{$$3$$}&{$$58$$}&{$$21.81$$}\\
{$$4$$}&{$$60$$}&{$$21.18$$}\\
{$$5$$}&{$$64$$}&{$$20.53$$}\\
{$$6$$}&{$$62$$}&{$$19.52$$}\\
{$$7$$}&{$$65$$}&{$$18.57$$}\\
{$$8$$}&{$$64$$}&{$$17.73$$}\\
{$$9$$}&{$$69$$}&{$$16.94$$}\\
{$$10$$}&{$$71$$}&{$$15.97$$}\\
{$$11$$}&{$$53$$}&{$$15.49$$}\\
{$$12$$}&{$$56$$}&{$$14.2$$}\\
{$$13$$}&{$$38$$}&{$$13.68$$}\\
\midrule
{$$\diameter$$}&{$$58.21\pm2.55$$}&{$$18.12\pm0.71$$}\\
\bottomrule
\end{tabular}
\end{table}

      \begin{table}[t]
\centering
\caption{Daten aus:rot mit magnet 90 (2).}
\label{tab:rot mit magnet 90 (2)}
\sisetup{table-format=1.2}
\begin{tabular}{S[table-format=3.2] S S S [table-format=3.2]}
\toprule
{Linie Nr.} & {Breite/[$\si[]{px}$]}&{Helligkeit}\\
\midrule
{$$0$$}&{$$40$$}&{$$17.05$$}\\
{$$1$$}&{$$57$$}&{$$19.98$$}\\
{$$2$$}&{$$58$$}&{$$20.95$$}\\
{$$3$$}&{$$58$$}&{$$21.81$$}\\
{$$4$$}&{$$60$$}&{$$21.18$$}\\
{$$5$$}&{$$64$$}&{$$20.53$$}\\
{$$6$$}&{$$62$$}&{$$19.52$$}\\
{$$7$$}&{$$65$$}&{$$18.57$$}\\
{$$8$$}&{$$64$$}&{$$17.73$$}\\
{$$9$$}&{$$69$$}&{$$16.94$$}\\
{$$10$$}&{$$71$$}&{$$15.97$$}\\
{$$11$$}&{$$53$$}&{$$15.49$$}\\
{$$12$$}&{$$56$$}&{$$14.2$$}\\
{$$13$$}&{$$38$$}&{$$13.68$$}\\
\midrule
{$$\diameter$$}&{$$58.21\pm2.55$$}&{$$18.12\pm0.71$$}\\
\bottomrule
\end{tabular}
\end{table}

      \begin{table}[t]
\centering
\caption{Daten aus:rot mit magnet 90 (2).}
\label{tab:rot mit magnet 90 (2)}
\sisetup{table-format=1.2}
\begin{tabular}{S[table-format=3.2] S S S [table-format=3.2]}
\toprule
{Linie Nr.} & {Breite/[$\si[]{px}$]}&{Helligkeit}\\
\midrule
{$$0$$}&{$$40$$}&{$$17.05$$}\\
{$$1$$}&{$$57$$}&{$$19.98$$}\\
{$$2$$}&{$$58$$}&{$$20.95$$}\\
{$$3$$}&{$$58$$}&{$$21.81$$}\\
{$$4$$}&{$$60$$}&{$$21.18$$}\\
{$$5$$}&{$$64$$}&{$$20.53$$}\\
{$$6$$}&{$$62$$}&{$$19.52$$}\\
{$$7$$}&{$$65$$}&{$$18.57$$}\\
{$$8$$}&{$$64$$}&{$$17.73$$}\\
{$$9$$}&{$$69$$}&{$$16.94$$}\\
{$$10$$}&{$$71$$}&{$$15.97$$}\\
{$$11$$}&{$$53$$}&{$$15.49$$}\\
{$$12$$}&{$$56$$}&{$$14.2$$}\\
{$$13$$}&{$$38$$}&{$$13.68$$}\\
\midrule
{$$\diameter$$}&{$$58.21\pm2.55$$}&{$$18.12\pm0.71$$}\\
\bottomrule
\end{tabular}
\end{table}

      \begin{table}[t]
\centering
\caption{Daten aus:rot mit magnet 90 (2).}
\label{tab:rot mit magnet 90 (2)}
\sisetup{table-format=1.2}
\begin{tabular}{S[table-format=3.2] S S S [table-format=3.2]}
\toprule
{Linie Nr.} & {Breite/[$\si[]{px}$]}&{Helligkeit}\\
\midrule
{$$0$$}&{$$40$$}&{$$17.05$$}\\
{$$1$$}&{$$57$$}&{$$19.98$$}\\
{$$2$$}&{$$58$$}&{$$20.95$$}\\
{$$3$$}&{$$58$$}&{$$21.81$$}\\
{$$4$$}&{$$60$$}&{$$21.18$$}\\
{$$5$$}&{$$64$$}&{$$20.53$$}\\
{$$6$$}&{$$62$$}&{$$19.52$$}\\
{$$7$$}&{$$65$$}&{$$18.57$$}\\
{$$8$$}&{$$64$$}&{$$17.73$$}\\
{$$9$$}&{$$69$$}&{$$16.94$$}\\
{$$10$$}&{$$71$$}&{$$15.97$$}\\
{$$11$$}&{$$53$$}&{$$15.49$$}\\
{$$12$$}&{$$56$$}&{$$14.2$$}\\
{$$13$$}&{$$38$$}&{$$13.68$$}\\
\midrule
{$$\diameter$$}&{$$58.21\pm2.55$$}&{$$18.12\pm0.71$$}\\
\bottomrule
\end{tabular}
\end{table}

      \begin{table}[t]
\centering
\caption{Daten aus:rot mit magnet 90 (2).}
\label{tab:rot mit magnet 90 (2)}
\sisetup{table-format=1.2}
\begin{tabular}{S[table-format=3.2] S S S [table-format=3.2]}
\toprule
{Linie Nr.} & {Breite/[$\si[]{px}$]}&{Helligkeit}\\
\midrule
{$$0$$}&{$$40$$}&{$$17.05$$}\\
{$$1$$}&{$$57$$}&{$$19.98$$}\\
{$$2$$}&{$$58$$}&{$$20.95$$}\\
{$$3$$}&{$$58$$}&{$$21.81$$}\\
{$$4$$}&{$$60$$}&{$$21.18$$}\\
{$$5$$}&{$$64$$}&{$$20.53$$}\\
{$$6$$}&{$$62$$}&{$$19.52$$}\\
{$$7$$}&{$$65$$}&{$$18.57$$}\\
{$$8$$}&{$$64$$}&{$$17.73$$}\\
{$$9$$}&{$$69$$}&{$$16.94$$}\\
{$$10$$}&{$$71$$}&{$$15.97$$}\\
{$$11$$}&{$$53$$}&{$$15.49$$}\\
{$$12$$}&{$$56$$}&{$$14.2$$}\\
{$$13$$}&{$$38$$}&{$$13.68$$}\\
\midrule
{$$\diameter$$}&{$$58.21\pm2.55$$}&{$$18.12\pm0.71$$}\\
\bottomrule
\end{tabular}
\end{table}

      \begin{table}[t]
\centering
\caption{Daten aus:rot mit magnet 90 (2).}
\label{tab:rot mit magnet 90 (2)}
\sisetup{table-format=1.2}
\begin{tabular}{S[table-format=3.2] S S S [table-format=3.2]}
\toprule
{Linie Nr.} & {Breite/[$\si[]{px}$]}&{Helligkeit}\\
\midrule
{$$0$$}&{$$40$$}&{$$17.05$$}\\
{$$1$$}&{$$57$$}&{$$19.98$$}\\
{$$2$$}&{$$58$$}&{$$20.95$$}\\
{$$3$$}&{$$58$$}&{$$21.81$$}\\
{$$4$$}&{$$60$$}&{$$21.18$$}\\
{$$5$$}&{$$64$$}&{$$20.53$$}\\
{$$6$$}&{$$62$$}&{$$19.52$$}\\
{$$7$$}&{$$65$$}&{$$18.57$$}\\
{$$8$$}&{$$64$$}&{$$17.73$$}\\
{$$9$$}&{$$69$$}&{$$16.94$$}\\
{$$10$$}&{$$71$$}&{$$15.97$$}\\
{$$11$$}&{$$53$$}&{$$15.49$$}\\
{$$12$$}&{$$56$$}&{$$14.2$$}\\
{$$13$$}&{$$38$$}&{$$13.68$$}\\
\midrule
{$$\diameter$$}&{$$58.21\pm2.55$$}&{$$18.12\pm0.71$$}\\
\bottomrule
\end{tabular}
\end{table}

      \begin{table}[t]
\centering
\caption{Daten aus:rot mit magnet 90 (2).}
\label{tab:rot mit magnet 90 (2)}
\sisetup{table-format=1.2}
\begin{tabular}{S[table-format=3.2] S S S [table-format=3.2]}
\toprule
{Linie Nr.} & {Breite/[$\si[]{px}$]}&{Helligkeit}\\
\midrule
{$$0$$}&{$$40$$}&{$$17.05$$}\\
{$$1$$}&{$$57$$}&{$$19.98$$}\\
{$$2$$}&{$$58$$}&{$$20.95$$}\\
{$$3$$}&{$$58$$}&{$$21.81$$}\\
{$$4$$}&{$$60$$}&{$$21.18$$}\\
{$$5$$}&{$$64$$}&{$$20.53$$}\\
{$$6$$}&{$$62$$}&{$$19.52$$}\\
{$$7$$}&{$$65$$}&{$$18.57$$}\\
{$$8$$}&{$$64$$}&{$$17.73$$}\\
{$$9$$}&{$$69$$}&{$$16.94$$}\\
{$$10$$}&{$$71$$}&{$$15.97$$}\\
{$$11$$}&{$$53$$}&{$$15.49$$}\\
{$$12$$}&{$$56$$}&{$$14.2$$}\\
{$$13$$}&{$$38$$}&{$$13.68$$}\\
\midrule
{$$\diameter$$}&{$$58.21\pm2.55$$}&{$$18.12\pm0.71$$}\\
\bottomrule
\end{tabular}
\end{table}

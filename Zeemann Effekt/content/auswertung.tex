


\section{Auswertung}
\label{sec:auswertung}

In diesem Kapitel werden die aufgenommenen Messwerte ausgewertet.
\subsection{Vorbereitung}
\label{sec:vorbereitunf}
\subsubsection{Lumer-Gehrcke Platte}
\label{sec:luhmer}
Die Eigenschaften der Lumer-Ghercke, Platte lassen sich mit den Materialeigenschaften
der Platte bestimmen. Die in diesem Versuch verwendete Lumer-Ghercke Platte hat die
Maße $d=\SI[]{4}[]{mm}$, $L=\SI[]{120}[]{mm}$. Die beiden Spektrallinien welche betrachtet werden sollen
sind:
\begin{center}
  $\lambda_{rot}=\SI[]{643.8}[]{nm}$ und
  $\lambda_{blau}=\SI[]{480.0}[]{nm}$.
\end{center}
Für diesen Versuchsaufbau ergeben sich die wellenlängenabhängigen Brechungsindizes:
\begin{center}
  $n_{rot}=1.4567$ und
  $n_{blau}=1.4635$.
\end{center}
Mit diesen Angaben kann dann über \autoref{eq:aufloesungsvermoegen} das Auflösungsvermögen $A$ und über
\autoref{eq:dispersionsgebiet} das Dispersionsgebiet $\symup{\Delta}\lambda$ berechnet werden. Die Ergebnisse
sind in \autoref{tab:LGPlatte} dargestellt.
\begin{table}
  \centering
    \caption{Wellenlängenabhängige Werte der Lummer-Gehrcke Platte, für Auflösungsvermögen und Dispersionsgebiet.}
    \label{tab:LGPlatte}
    \sisetup{table-format=1.2}
    \begin{tabular}{S[table-format=3.2] | S S S [table-format=3.2]}
      \toprule
      {Größe}&{$\SI[]{643.8}[]{nm}$} & {$\SI[]{480.0}[]{nm}$}\\
      \midrule
      {$$A$$}&{$$209128.59$$}&{$$285458.06$$}\\
      {$\symup{\Delta} \lambda_D / \si[]{\pico\metre} $}&{$$48.91$$}&{$$26.95$$}\\
      \bottomrule
    \end{tabular}
  \end{table}

  \subsubsection{Bestimmung der Landé-Faktoren}
  Die Unterschiede in der Aufspaltung der Spektrallinien ist abhängig von den Landé
Faktoren $g_j$. Für den roten Übergang zwischen den Niveaus $^1 P_1 \leftrightarrow  ^1D_2$ und den blauen
Übergang zwischen den Niveaus $^3 S_1 \leftrightarrow  ^3P_1$ ergeben sich für die einzelnen Landé Faktoren,
mit \autoref{eq:lande} die Werte in \autoref{tab:lande}. Mit der Notation $^{2𝑆+}1𝐿_𝑗$ für die Niveaus, wobei
$L$ die Bezeichnung des Niveaus ist, welchem ein fester Drehimpuls zugeordnet ist, ergeben sich
\begin{table}
  \centering
    \caption{Berechnung der Landé-Faktoren.}
    \label{tab:lande}
    \sisetup{table-format=1.2}
    \begin{tabular}{S[table-format=3.2] | S S S S S [table-format=3.2]}
      \toprule
      {Niveau}&{J}&{S}&{L}  & {$g_j$}\\
      \midrule
      {$ ^1 P_1$}&{1}&{0}&{1}&{$1$}\\
      {$ ^1 D_2$}&{2}&{0}&{2}&{$1$}\\
      {$ ^3 S_1$}&{1}&{1}&{0}&{$2$}\\
      {$ ^3 P_1$}&{1}&{1}&{1}&{$\frac{2}{2}$}\\
      
      \bottomrule
    \end{tabular}
  \end{table}
  Für die Energiedifferenz zwischen den einzelnen Niveaus ergeben sich nun aus \autoref{eq:nzeeman} 
  folgend die Werte für den normalen Zeeman Effekt in \autoref{tab:normZeemann}.
  \begin{table}
    \centering
      \caption{Berechnung der Landé-Faktoren des normalen Zeemann-Effektes.}
      \label{tab:normZeemann}
      \sisetup{table-format=1.2}
      \begin{tabular}{S[table-format=3.2]  S S S S S [table-format=3.2]}
        \toprule
        {$\SI[]{643.8}[]{\nano \metre}$}&{$$\Delta m =-1$$}&{$$\Delta m =0$$}  & {$$\Delta m =+1$$}\\
        \midrule
        {$ $}&{$\mu_BB$}&{$0$}&{$\mu_BB$}\\
        \bottomrule
      \end{tabular}
    \end{table}
  Für den anormalen Zeeman Effekt können die Faktoren über \autoref{eq:anzeeman} bestimmt werden.
  Sie sind in \autoref{tab:anormZeemann} dargestellt.
    \begin{table}
      \centering
        \caption{Berechnung der Landé-Faktoren des anormalen Zeemann-Effektes.}
        \label{tab:anormZeemann}
        \sisetup{table-format=1.2}
        \begin{tabular}{S[table-format=3.2]  S S S S S [table-format=3.2]}
          \toprule
          {$\SI[]{480.0}[]{\nano \metre}$}&{$$\Delta m =-1$$}&{$$\Delta m =0$$}  & {$$\Delta m =+1$$}\\
          \midrule
          {$ $}&{$\frac{3}{2}\mu_BB$}&{$-\frac{1}{2}\mu_BB$}&{$--$}\\
          {$ $}&{$2\mu_BB$}&{$0$}&{$-2\mu_BB$}\\
          {$ $}&{$--$}&{$\frac{1}{2}\mu_BB$}&{$\frac{3}{2}\mu_BB$}\\
          \bottomrule
        \end{tabular}
      \end{table}
\subsubsection{Berechnung der optimalen B-Feldstärken}
Um zu vermeiden, dass sich Linien überschneiden da sie sich entweder zu weit voneinander entfernen 
oder nicht weit genug, werden hier die optimalen Feldstärken berechnet.
Das optimale B-Feld ergibt sich über:
\begin{equation}
  B=\frac{hc}{4\mu_B\lambda^2g_{ij}}=\frac{\Delta E}{\mu_B g_{ij}}
\end{equation}
Die sich daraus ergebenden idealen B-Feldstärken sind in \autoref{tab:feldstaerken} dargestellt.
\begin{table}
  \centering
    \caption{Optimale B-Feldstärken für die verschiedenen Landé-Faktoren.}
    \label{tab:feldstaerken}
    \sisetup{table-format=1.2}
    \begin{tabular}{S[table-format=3.2]  S S S S S [table-format=3.2]}
      \toprule
      {$\Delta g_{ij}$}&{$B$/T}\\
      \midrule
      {$0.5$}&{$1.25$}\\
      {$1.0$}&{$0.63$}\\
      {$1.5$}&{$0.42$}\\
      {$2.0$}&{$0.31$}\\
      \bottomrule
    \end{tabular}
  \end{table}
\subsection{Vermessung des Elektromagneten}
Es ist aufgrund des Versuchsaufbaus nicht möglich die magnetische Flussdichte $B$ zwischen den 
beiden Polschuhen des Elektromagneten zu bestimmen während die Cadmiumdampflampe eingeführt ist.
Daher muss das Magnetfeld vorher mittels einer Hallsonde in abhängigkeit vom Spulenstrom ausgemessen
werden. Wenn die Cadmiumdampflampe dann eingeführt ist muss nurnoch der passende Spulenstrom 
eingestellt werden. In \autoref{tab:magnetfeld} sind die Messdaten für die Neukurve dargestellt.
\begin{table}
    \centering
      \caption{In der Tabelle sind die Messdaten für den Spulenstrom $I$ und die resultierende Flussdichte $B$ dargestellt.}
      \label{tab:magnetfeld}
      \sisetup{table-format=1.2}
      \begin{tabular}{S[table-format=3.2] S S [table-format=3.2]}
        \toprule
        {$I$/[$\si[]{A}$]} & {$B$/[$\si[]{mT}$]}\\
        \midrule
        {$$5.0$$} & {$$ 452.1$$}\\
        {$$4.8$$} & {$$	440.4$$}\\
        {$$4.6$$} & {$$	430.2$$}\\
        {$$4.4$$} & {$$	415.4$$}\\
        {$$4.2$$} & {$$	403.4$$}\\
        {$$4.0$$} & {$$	388.0$$}\\
        {$$3.8$$} & {$$	371.7$$}\\
        {$$3.6$$} & {$$	356.7$$}\\
        {$$3.4$$} & {$$	338.8$$}\\
        {$$3.2$$} & {$$	320.9$$}\\
        {$$3.0$$} & {$$	305.5$$}\\
        {$$2.8$$} & {$$	288.2$$}\\
        {$$2.6$$} & {$$	266.8$$}\\
        {$$2.4$$} & {$$	248.8$$}\\
        {$$2.2$$} & {$$	229.6$$}\\
        {$$2.0$$} & {$$	209.2$$}\\
        {$$1.6$$} & {$$	169.8$$}\\
        {$$1.2$$} & {$$	131.1$$}\\
        {$$0.8$$} & {$$	89.4$$}\\
        {$$0.4$$} & {$$	50.7$$}\\
        {$$0.0$$} & {$$	 9.9$$}\\
        
        \bottomrule
      \end{tabular}
    \end{table}
\FloatBarrier
    Die Daten aus \autoref{tab:magnetfeld} wurden in \autoref{fig:magnetfeld} dargestellt. Zudem wurde 
    an die Daten ein Polynom dritten Grades angepasst die verwendeten Parameter lauten:
    \begin{center}
        $$a_3 = -0.00105 \pm 0.00008$$\\
        $$a_2 =  0.00359 \pm 0.00062$$\\
        $$a_1 =  0.09673 \pm 0.00135$$\\
        $$a_0 =  0.01059 \pm 0.00080$$
    \end{center}
    Das zugehörige Polynom hat die Form:
    \begin{equation*}
      B(I)=a_3I^3+a_2I^2+a_1I+a_0
    \end{equation*}

    \begin{figure}
        \centering
        \includegraphics[width=1\textwidth]{content/grafiken/magnetfeld.pdf}
        \caption{Magnetische Flussdichte des verwendeten Elektromagneten in Abhängigkeit des Spulenstroms.}
        \label{fig:magnetfeld}
      \end{figure}
\subsection{Vermessung der Spektrallinien}
\label{sec:linien}
Um die Linien zu vermessen wurde das Licht aus der Lummer-Gehrcke Platte mit einer CAD-Kamera
aufgenommen. Die aufgenommenen Bilder wurden dann händisch zugeschnitten. 
Anschließend wurde mittels eines Python-Programms jeweils die  vertikal mittlere
Pixelzeile herausgeschnitten und in Graustufen umgerechnet. Die jeweiligen Werte für die Helligkeit wurden 
gegen die Pixelposition aufgetragen, Die jeweiligen Plots sind in \autoref{sec:plots} zu sehen.  Das Programm 
zählte dann die  Pixel welche zwischen zwei Maxima liegen. 
\FloatBarrier
\subsubsection{Die rote Sigma-Linie}
\label{sec:rot}
Zunächst wird der normale Zeeman-Effekt an der roten $\sigma$-Linie der Cadmiumdampflampe vermessen.
In \autoref{fig:rot0} sind die Aufnahmen der Lummer-Gehrcke-Platte zu sehen. Dabei stand der Polarisationsfilter
bei $\phi=\SI[]{90}[]{\degree}$. Im oberen Teil der Grafik ist das Magnetfeld abgeschaltet, im unteren eingeschaltet.
Im direkten Vergleich ist sofort die Aufspaltung erkennbar.
\begin{figure}
  \centering
  \includegraphics[width=0.75\textwidth]{content/grafiken/rot 0.JPG}
  \caption{Die rote Spektrallienie, oben ohne Magnetfeld, unten mit Magnetfeld.}
  \label{fig:rot0}
\end{figure}

Aus den mit dem beschriebenen Programm berechneten Daten wurde dann die Wellenlängenverschiebung $\delta \lambda$
berechent. Sie ist zusammen mit den gefundenen Daten für $\delta s$ und $\Delta s$ in \autoref{tab:tabelleRot}
dargestellt. Die Wellenlängenverschiebung berechnet sich über:

\begin{equation}
  \delta \lambda =\frac{\delta s}{2\delta S}\Delta \lambda_D
\end{equation}
\begin{table}
\centering
\caption{Wellenlaengenverschiebung der roten Sigma-Linie.}
\label{tab:tabelleRot}
\sisetup{table-format=1.2}
\begin{tabular}{S[table-format=3.2] S S S [table-format=3.2]}
\toprule
{Mode Nr.} & {$\Delta S$/px}&{$\delta S$ /px}&{$\delta \lambda$}\\
\midrule
{$$0$$}&{$$119$$}&{$$37$$}&{$$107.6754$$}\\
{$$1$$}&{$$124$$}&{$$31$$}&{$$94.005$$}\\
{$$2$$}&{$$124$$}&{$$41$$}&{$$124.3292$$}\\
{$$3$$}&{$$133$$}&{$$52$$}&{$$169.1308$$}\\
{$$4$$}&{$$134$$}&{$$47$$}&{$$154.0176$$}\\
{$$5$$}&{$$145$$}&{$$53$$}&{$$187.9367$$}\\
{$$7$$}&{$$149$$}&{$$44$$}&{$$160.327$$}\\
{$$8$$}&{$$160$$}&{$$56$$}&{$$219.1168$$}\\
{$$9$$}&{$$165$$}&{$$48$$}&{$$193.6836$$}\\
{$$10$$}&{$$180$$}&{$$71$$}&{$$312.5349$$}\\
{$$11$$}&{$$196$$}&{$$81$$}&{$$388.2476$$}\\
{$$12$$}&{$$218$$}&{$$75$$}&{$$399.8392$$}\\
\midrule
{$$\diameter$$}&{$$$$}&{$$$$}&{$$209.24\pm29.86$$}\\
\bottomrule
\end{tabular}
\end{table}

Aus diesen Daten wurde dann der Mittelwert nach \autoref{eq:Mittelwert} mit zugehörigem Fehler nach
\autoref{eq:mittelwertfehler} berechent.

\FloatBarrier
\newpage
\subsubsection{Die blaue Pi-Linie}
Um die blaue $\pi$-Linie des annormalen Zeemaneffektes beobachten zu können muss der 
Aufbau des Experiments geringfügig verändert werden. Als erstes wird der Spalt des optischen 
Aufbaus so eingestellt, dass nun die blaue Spektrallienie auf die Lummer-Gehrcke-Platte trifft.
Anschließend wird der Polarisationsfilter auf $\phi=\SI[]{90}[]{\degree}$ eingestellt.
Dann kann analog zu \autoref{sec:rot} vorgegeangen werden. Die dabei errechnten Werte
sind in \autoref{tab:tabelleBlau2} dargestellt. In \autoref{fig:blau90} ist der direkte Vergleich
der $\pi$-Linie mit ein- und ausgeschaltetem Magnetfeld zu sehen.
\begin{figure}
  \centering
  \includegraphics[width=0.75\textwidth]{content/grafiken/blau 90.JPG}
  \caption{Die Pi-Linie, oben ohne Magnetfeld, unten mit Magnetfeld.}
  \label{fig:blau90}
\end{figure}
\begin{table}
\centering
\caption{Wellenlängenverschiebung der blauen Pi-Linie.}
\label{tab:tabelleBlau2}
\sisetup{table-format=1.2}
\begin{tabular}{S[table-format=3.2] S S S [table-format=3.2]}
\toprule
{Mode Nr.} & {$\Delta S$/px}&{$\delta S$ /px}&{$\delta \lambda$}\\
\midrule
{$$0$$}&{$$38$$}&{$$39$$}&{$$19.97$$}\\
{$$1$$}&{$$38$$}&{$$37$$}&{$$18.9458$$}\\
{$$2$$}&{$$39$$}&{$$36$$}&{$$18.9189$$}\\
{$$4$$}&{$$39$$}&{$$36$$}&{$$18.9189$$}\\
{$$5$$}&{$$35$$}&{$$38$$}&{$$17.9218$$}\\
\midrule
{$$\diameter$$}&{$$$$}&{$$$$}&{$$18.94\pm0.32$$}\\
\bottomrule
\end{tabular}
\end{table}



\FloatBarrier
\subsubsection{Die blaue Sigma-Linie}
Für die blaue $\sigma$-Linie wurden exakt analog zu  \autoref{sec:rot} vorgegeangen. Die Aufspaltung
der Linien ist in \autoref{fig:blau0} zu sehen. Die Daten sind zusammen mit ihrem Mittelwert in \autoref{tab:tabelleBlau}
dargestellt.
\begin{figure}
  \centering
  \includegraphics[width=0.75\textwidth]{content/grafiken/blau 90.JPG}
  \caption{Die blaue Sigma-Linie, oben ohne Magnetfeld, unten mit Magnetfeld.}
  \label{fig:blau0}
\end{figure}
\begin{table}
\centering
\caption{Wellenlängenverschiebung der blauen Sigma-Linie.}
\label{tab:tabelleBlau}
\sisetup{table-format=1.2}
\begin{tabular}{S[table-format=3.2] S S S [table-format=3.2]}
\toprule
{Mode Nr.} & {$\Delta S$/px}&{$\delta S$ /px}&{$\delta \lambda$}\\
\midrule
{$$0$$}&{$$33$$}&{$$22$$}&{$$9.7828$$}\\
{$$1$$}&{$$35$$}&{$$23$$}&{$$10.8474$$}\\
{$$2$$}&{$$36$$}&{$$24$$}&{$$11.6424$$}\\
{$$3$$}&{$$35$$}&{$$23$$}&{$$10.8474$$}\\
{$$4$$}&{$$36$$}&{$$22$$}&{$$10.6722$$}\\
{$$6$$}&{$$37$$}&{$$22$$}&{$$10.9686$$}\\
{$$7$$}&{$$36$$}&{$$23$$}&{$$11.1573$$}\\
{$$8$$}&{$$37$$}&{$$23$$}&{$$11.4672$$}\\
{$$9$$}&{$$36$$}&{$$23$$}&{$$11.1573$$}\\
{$$10$$}&{$$37$$}&{$$21$$}&{$$10.4701$$}\\
\midrule
{$$\diameter$$}&{$$$$}&{$$$$}&{$$10.9\pm0.17$$}\\
\bottomrule
\end{tabular}
\end{table}



\FloatBarrier
\subsection{Bestimmung der Landé-Faktoren}
Um die Landé Faktoren zu bestimmen, wird von der Formel für die Veränderung der
Energie \autoref{eq:anzeeman} ausgegangen:
\begin{equation*}
  \Delta E = g_{ij} \mu_B B m_J\\
  \Leftrightarrow g_{ij}=\frac{\Delta E}{\mu_B B} 
\end{equation*}
$\mu_B$ ist dabei das Bohrsche Magneton und $g=g_{ij}$ der gewünschte Landé Faktor, nach
dem direkt umgestellt wurde. In erster Näherung
\begin{equation*}
  \Delta E=E(\lambda+\delta \lambda) - E(\lambda)
\end{equation*}
ergibt die Taylorentwicklung:
\begin{equation*}
  \Delta E=\frac{\delta E}{\delta \lambda} E(\lambda).
\end{equation*}
Diese wird mit der quantenmechanischen Energie $E(\lambda)=\frac{hc}{\lambda }$
in den Landé Faktor eingesetzt und es folgt:
\begin{equation*}
  g=\frac{\delta \lambda hc}{\mu_B B \lambda^2}
\end{equation*}
Nun können mithilfe der in \autoref{sec:linien} bestimmten Wellenlängenverschiebungen die
Landé-Faktoren berechnet werden.


\begin{table}
  \centering
    \caption{Die berechneten Landé-Faktoren}
    \sisetup{table-format=1.2}
    \begin{tabular}{S[table-format=3.2] S S S S [table-format=3.2]}
      \toprule
      {$\lambda/\si[]{\nano \metre}$} &{Übergang} & {$\delta \lambda / \si[]{\pico \metre}$}&{$g_{j}$}\\
      \midrule
      {$648.8$}&{$\sigma$}&{$209.24\pm 29.86$}&{$23.5504\pm $}\\
      {$480.0$}&{$\sigma$}&{$8.89\pm 0.7$}&{$2.3817\pm $}\\
      {$480.0$}&{$\pi$}&{$21.34\pm 0.44 $}&{$4.3882\pm $}\\
      \bottomrule
    \end{tabular}
  \end{table}





\FloatBarrier






  \subsection{Helligkeitsplots}
\label{sec:plots}
\twocolumn[]
      
      \newpage
      \begin{figure}
        \centering
        \includegraphics[width=0.5\textwidth]{content/grafiken/blau mit magnet 0 gimbplot.pdf}
        \caption{}
        \label{fig:bmm0}
      \end{figure}
      
      \begin{figure}
        \centering
        \includegraphics[width=0.5\textwidth]{content/grafiken/blau mit magnet 90 gimbplot.pdf}
        \caption{}
        \label{fig:bmm90}
      \end{figure}
      
      \begin{figure}
        \centering
        \includegraphics[width=0.5\textwidth]{content/grafiken/blau ohne magnet 0 gimbplot.pdf}
        \caption{}
        \label{fig:bom0}
      \end{figure}
      
      \begin{figure}
        \centering
        \includegraphics[width=0.5\textwidth]{content/grafiken/blau ohne magnet 90 gimbplot.pdf}
        \caption{}
        \label{fig:bom90}
      \end{figure}
      
      \begin{figure}
        \centering
        \includegraphics[width=0.5\textwidth]{content/grafiken/rot mit magnet 0 gimbplot.pdf}
        \caption{}
        \label{fig:rmm0}
      \end{figure}
      
      \begin{figure}
        \centering
        \includegraphics[width=0.5\textwidth]{content/grafiken/rot mit magnet 90 gimbplot.pdf}
        \caption{}
        \label{fig:rmm90}
      \end{figure}
      
      \begin{figure}
        \centering
        \includegraphics[width=0.5\textwidth]{content/grafiken/rot ohne magnet 0 gimbplot.pdf}
        \caption{}
        \label{fig:rom0}
      \end{figure}
      
      \begin{figure}
        \centering
        \includegraphics[width=0.5\textwidth]{content/grafiken/rot ohne magnet 90 gimbplot.pdf}
        \caption{}
        \label{fig:rom90}
      \end{figure}
\onecolumn


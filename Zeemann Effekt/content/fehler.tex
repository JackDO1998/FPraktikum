\section{Fehler}
\label{sec:Fehler}
Der Mittelwert:
\begin{center}
  \begin{equation}
    \label{eq:Mittelwert}
  \bar{x} = \frac{1}{n} \sum \nolimits_{i=0} x_i
  \end{equation} 
\end{center}

Die Standardabweichung:
\begin{center}
  \begin{equation}
    \label{eq:standardabweichung}
    \sigma=\sqrt{\frac{\sum(x_i-\bar{x})^2}{n-1}}
  \end{equation}
\end{center}

Der Fehler des Mittelwertes:
\begin{center}
  \begin{equation}
    \label{eq:mittelwertfehler}
    \sigma_{\bar{x}}=\frac{\sigma}{\sqrt{n}}
  \end{equation}

  
\end{center}

%Die Poissonverteilung:
%\begin{center}
%    \begin{equation}
%        \label{eq:poisson}
%        \Delta N=\sqrt{N}
%        \end{equation}
%\end{center}

Die Gaußsche Fehlerfortpflanzung:
\begin{center}
\begin{equation}
  \label{eq:gaussfehler}  
\sigma_x=\sqrt{(\frac{\partial f}{\partial x_1})^2\sigma_{x_1}^2+(\frac{\partial f}{\partial x_2})^2\sigma_{x_2}^2+...+(\frac{\partial f}{\partial x_n})^2\sigma_{x_n}^2}
\end{equation}
\end{center}

Die Prozentuale Abweichung:

\begin{center}
  \begin{equation}
    \label{eq:prozentuale} 
    Abweichung=\frac{Experimenteller Wert - Theoriewert}{Theoriewert}\times 100 
   \end{equation}
  \end{center}
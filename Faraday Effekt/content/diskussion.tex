

\section{Diskussion}
\label{sec:Diskussion}

Dieses Kapitel befasst sich mit der Diskussion der im \autoref{sec:auswertung} erhaltenen Ergebnisse.
Für die maximale Magnetflussdichte ließ sich der Wert zu $\text{B} = \SI{412}{\milli\tesla}$ ablesen.
Eine genauere Bestimmung wäre durch mehrfaches Messen und dann fitten eines Polynoms zweiten Grades an die Messdaten in unmittelbarer Nähe des Maximums möglich gewesen.
Dies ist jedoch im Rahmen der Genauigkeit nicht nötig, da andere Faktoren die Unsicherheit des Wertes maßgeblich beeinflussen.
Zum einen ist die Annahme, dass am Ort der Probe das Magnetfeld seine maximale Flussdichte annimmt möglicherweise nicht ganz richtig.
Zum anderen erhitzt sich der Elektromagnet während des Versuchs stetig, was zu Änderungen der Magnetflussdichte führen könnte.\\

Laut \autoref{eq:nichtfrei} soll der Rotationswinkel je Einheitslänge der hochreinen Probe eine proportionalität zu $\lambda^{-2}$ aufweisen.
Dies ist in \autoref{fig:probe3} sehr gut erkennbar.
Die gemessenen Werte zur hochreinen Probe stimmen also sichtbar gut mit der Theorie überein.\\
Auch in \autoref{fig:probe1leit} und \autoref{fig:probe2leit} sind die Winkel gegen $\lambda^2$ aufgetragen.
Hier sollten die Messwerte als linearer Verlauf zu sehen sein.
Sie weichen in beiden Diagrammen vom Fit der Funktion \autoref{eq:fit} an die Messwerte ab, wobei in \autoref{fig:probe1leit} deutlich stärkere Schwankungen erkennbar sind.\\
Mögliche Gründe könnten die zuvor erwähnte Unsicherheit des Magnetfeldes sein, aber wahrscheinlicher sorgt das Zusammenspiel der Messinstrumente für größere Unsicherheiten.
Auch Fehler bei der Justage oder unsachgemäßes Ablesen der Winkel könnten zu Unsicherheiten beitragen.\\

Für die effektive Masse ergaben sich die Werte:
\begin{equation*} \label{eq:efmas}
\begin{split}
m^\star_\text{Probe1}& = \SI{0,085\pm0,028}{m_e}\\
m^\star_\text{Probe2}& = \SI{0,068\pm0,008}{m_e}
\end{split}
\end{equation*}
Für die erste Probe weicht der Wert um 35\% vom Literaturwert\footnote{siehe http://www.ioffe.ru/SVA/NSM/Semicond/GaAs/basic.html} ab.
Der Wert für die zweite Probe hat eine Abweichung von 8\%.
Beide Werte stimmen trotz starker Abweichungen der Rotationswinkel vom theoretischen Verlauf gut mit der Literatur überein, wobei die errechnete effektive Masse der zweiten Probe eine sehr gute Übereinstimmung aufweist.

















\section{Literatur}
\label{sec:literatur}
$[$1$]$ TU Dortmund. Versuchsanleitung zu Versuch V46: Faraday-Effekt an Halbleitern.\\
$[$2$]$ TU Dortmund. Anhang 1, V46 - Faraday-Effekt an Halbleitern.\\
$[$3$]$ Wolfgang Demtröder. Experimentalphysik 3. Atome, Moleküle und Festkörper. 4. Aufl. Springer-Verlag Berlin, 2010.\\
$[$4$]$ http://www.ioffe.ru/index_en.html\\
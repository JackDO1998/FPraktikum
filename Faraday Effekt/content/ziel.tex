\section{Zielsetzung}
\label{sec:ziel}
Eine elektromagnetische Welle, welche linear polarisiert ist, kann in einem für die Welle durchlässigen Medium,
welches von einem parallel zur Wellenausbreitungsrichtung gerichteten Magnetfeldes durchströmt wird, eine Drehung
der Polarisationsebene um die Achse der Ausbreitungsrichtung erfahren. Dieser Vorgang wird als Faradayeffekt 
bezeichnet. Mithelfe diese Effektes soll im folgenden die effektive Masse $m^\star$ von Elektronen im Leitungsband
von verschiedenen n-dotierten Proben des Halbleiters Galliumarsenid bestimmt werden.
